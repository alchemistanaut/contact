% Copyright 2017 Justin Gombos
% 
% Licensed under the Apache License, Version 2.0 (the "License");
% you may not use this file except in compliance with the License.
% You may obtain a copy of the License at
% 
%     http://www.apache.org/licenses/LICENSE-2.0
% 
% Unless required by applicable law or agreed to in writing, software
% distributed under the License is distributed on an "AS IS" BASIS,
% WITHOUT WARRANTIES OR CONDITIONS OF ANY KIND, either express or implied.
% See the License for the specific language governing permissions and
% limitations under the License.

\documentclass[pdftex,12pt,titlepage=false]{scrartcl}

\usepackage[svgnames]{xcolor} %for LightGoldenrodYellow
\usepackage[margin=7mm,bottom=10mm,pdftex,letterpaper]{geometry}
\usepackage[utf8]{inputenc}
\usepackage[T1]{fontenc} %suggested to avoid ``OT1 encoding''
\usepackage{hyperref}
\usepackage{array}
\usepackage[pdftex]{graphicx}
\usepackage{multicol}
\usepackage{wrapfig}

\title{\rmfamily Configuring Outlook's built-in Cryptosystem}
%\newcommand{\theauthor}{J.G.}
%\author{\rmfamily\theauthor}
\date{\rmfamily\today}

\newcommand{\comment}[1]{} %inline comment by gobbling the argument

\newcommand{\secorio}{\href{https://www.secorio.com/}{\includegraphics[width=2cm]{images/logo_comodo.png}\tiny via Secorio}}

\begin{document}

\maketitle

\tableofcontents

\section{Prerequisites}
\begin{itemize}
\item Workstation with a Mozilla-based browser (e.g. Firefox) or
  Internet Explorer installed.  IE may be simpler because it seems to
  automatically store the key where Outlook needs it.  Chrome/Chromium
  will not work because it cannot handle key generation (as of the
  time of this composition).
\item MS Outlook 2013 or later (earlier versions officially support
  S/MIME but users often report difficulties, particularly with
  Outlook 2010)
\end{itemize}

\section{Prep to receive encrypted mail or to send signed mail}
\subsection{YouTube video (alternative to this document)}
For Outlook 2013 and 2016 users there is a comprehensive YouTube video
demonstrating whole process using Comodo for the certificate
authority.  It starts with nine minutes of blather, but {\large
  \href{https://www.youtube.com/watch?v=sfancZGEGjg\&start=535}{this
    link}} skips straight to the relevant part.  That video is
detailed enough to replace this entire document.

\subsection{Get an S/MIME certificate}\label{catable}
In your browser go to one of these certificate authorities (\textbf{Justin recommends Comodo via Secorio}):\\

\begin{tabular}{lp{2.3cm}l>{\tiny}l}
  \sl certificate authority (``CA'')& \sl price \newline\tiny(for non-commercial individual use) & \sl validity & \sl\normalsize notes\\
  \hline\\
  \href{https://www.cacert.org/}{\includegraphics[width=2cm]{images/logo_cacert4.png}} & gratis & 6|24 mos.\tiny (\href{http://wiki.cacert.org/FAQ/Privileges}{criteria}) & community driven\\
  \href{https://secure.comodo.com/products/frontpage?area=SecureEmailCertificate}{\includegraphics[width=2cm]{images/logo_comodo.png}\tiny direct} & gratis & 1 yr &\\
  \href{https://www.instantssl.com/ssl-certificate-products/free-email-certificate.html}{\includegraphics[width=2cm]{images/logo_comodo.png}\tiny via InstantSSL} & gratis & 1 yr &\\
  \secorio & gratis & 1 yr & recommended; assumed choice by this guide\\
  \href{https://www.entrust.com/secure-email-certificates/}{\includegraphics[width=2cm]{images/logo_entrust.png}} & $\geq$\$20 & \\
  \href{https://www.identrust.com/certificates/trustid.html}{\includegraphics[width=2cm]{images/logo_trustid.png}} & $\geq$\$19 & \\
  \href{https://www.startcomca.com/}{\includegraphics[width=2cm]{images/logo_startcom.png}} & gratis & 2 yrs & \\
  \href{https://buy.wosign.com/free/}{wosign} & gratis? & 2 yrs? & blocks tor?\\
\end{tabular}\\

  % geotrust, symantec, thawte have discontinued service
{\tiny Warning: the CAs that participate in e-mail certificate
  verification are constantly changing.  Many CAs have discontinued
  e-mail certification prior to this guide.  Those are obviously
  omitted here, but some of the above listings are likely to become
  obsolete as this guide ages.  Consequently it might be interesting
  to check out the catalog of certificate authorities listed at
  \url{http://kb.mozillazine.org/Getting_an_SMIME_certificate}).}

\subsubsection{If you chose ``Comodo via Secorio''}
\begin{multicols}{2}
  \begin{enumerate}
  \item (secorio.com) If you are using the \emph{noscript} firefox
    plugin, you must enable javascript for \texttt{secorio.com} and
    \texttt{comodo.com}.
  \item (secorio.com) In the left frame, select ``\texttt{S/MIME Class
      2}'' (even though it's \emph{class 1} that we need), then click
    ``\texttt{Order}''.
  \item (secorio.com) Scroll down to ``\texttt{S/MIME Certificates}''
    and choose ``1 year'' in the pull-down to the right of the
    \emph{class 1} row.
  \item (comodo.com) Fill out the form that appears in a new tab.
    Setting a revocation password is optional (and it's a good
    idea).% Untick the
    % ``Comodo newsletter opt in'' crap (if it's there).
  \item (your inbox) An e-mail will arrive.  If your e-mail client
    renders it graphically, click the button ``\texttt{Click \&
      Install Comodo Email Certificate}''.  For text clients, follow
    the instructions in the e-mail.  If your mail client does not
    automatically use Firefox or IE to open URLs, right-click that
    button instead, copy the URL, and paste it in the address bar to
    force it to render in Firefox or IE.
  \item Skip to section~\ref{browser_export}
  \end{enumerate}
\end{multicols}

\subsubsection{If you chose another certificate authority}
Simply follow the instructions on the website of the CA.  It will
generally involve filling out a form and confirming an e-mail.

\subsection{Installing your certificate into Outlook}\label{browser_export}

Generally you will follow
\href{https://www.ablebits.com/office-addins-blog/2014/04/11/email-encryption-outlook/}{this
  document}.  Ignore the top portion because you already have a
``digital ID''.  Scroll down to ``How to set up your e-mail
certificate in Outlook''.  That will configure Outlook for using your
certificate.

That process assumes Outlook already has your key automatically, and
thus no export/import are needed.  If there are no issues with the key
assignment step (that is, you were able to find your Comodo key), then
you can skip the rest of the section.

\subsubsection{If the key was not in the Trust Center..}
Some Outlook users have reported that in their environment (Windows
and Outlook versions) the key is not automatically visible in Outlook.
\textbf{If your Comodo key was not found} in the Outlook ``Trust
Center'', then watch
\href{https://www.youtube.com/watch?v=wGHaB0elkaA}{this YouTube video}
or follow the steps below to export the key from the browser and then
import it into Outlook.  That video demonstrates using Outlook 2010
for the mail client and Firefox for the browser.  Ignore the beginning
segment about using Symantec to create a key (Symantec no longer
offers the service; Comodo is recommended).

These steps assume Firefox was used for the key creation:

\begin{enumerate}
  \item (Firefox) go to: menu ($\equiv$) $\gg$ Options/Preferences $\gg$ Advanced
  $\gg$ Certificates $\gg$ View Certificates $\gg$ Your Certificates.\\[1em]%
  \includegraphics[width=0.7\textwidth]{images/firefox_cert_settings.png}
\item Highlight the line showing your new key.  It will be under the
  name of the CA you chose (e.g. the line under ``COMODO CA Limited''
  if you chose Comodo).
\item %\raisebox{0.6\baselineskip}{\parbox[t]{0.9\textwidth}{%
      %\begin{wrapfigure}{r}{0.7\textwidth}%
      %  \includegraphics[width=0.7\textwidth]{mailapp_firefox_cert_settings.png}
      %\end{wrapfigure}
  Click ``\texttt{Backup...}'' to export the key.
\item Save the file somewhere with a filename of your choice.  It will
  likely be given a \verb|.p12| extension.
\item\label{makebupw} You will be prompted for a password for the
  backup file.  A weak password is fine, because this backup file will
  not be transmitted or retained for long.  You will import it into
  Outlook locally, and then you will delete the backup file.
\item Now your private key must be imported into Outlook.  I'm not
  entirely sure how to do it, but
  \href{https://www.ablebits.com/office-addins-blog/2014/04/11/email-encryption-outlook/}{this
    document} gives the steps for configuring Outlook as
  needed. Ignore the top of the document because you already have a
  ``digital ID'', and scroll down to ``How to set up your e-mail
  certificate in Outlook''.  This will take you to the ``Trust
  Center'', the place in the settings where you can import the key
  from the backup file.
\item After the key is imported into Outlook, you should delete the
  backup file.  (You can always create a new backup file from Firefox
  if needed).
\end{enumerate}

\subsection{Distribute your S/MIME certificate (aka public key)}
In short: simply send an e-mail to the recipient using Outlook, and sign the message.\\

Detailed explanation: You have a pair of keys (these were created in
section~\ref{catable}).  One is a public key and the other is a
private key.  The public key must be sent to those who will send you
encrypted e-mail.  They will use your public key to encrypt messages
to you.  Your public key is automatically contained in the signature
of all messages you sign.

So to distribute your public key, simply send the other party an
signed e-mail from Outlook.  Encrypting this key distribution message
is optional, but it must be signed.  They can then extract your public
key from your signature.

\section{Prep to send encrypted mail% or verify signatures on received mail
}
Before you can send someone an encrypted message, you need the
recipients S/MIME certificate (public key).  This will normally come
to you when they send you a signed message, at which point you can
extract the certificate.  The certificate must then be associated to
that person in your address book.

\end{document}

