% Copyright 2017 Justin Gombos
% 
% Licensed under the Apache License, Version 2.0 (the "License");
% you may not use this file except in compliance with the License.
% You may obtain a copy of the License at
% 
%     http://www.apache.org/licenses/LICENSE-2.0
% 
% Unless required by applicable law or agreed to in writing, software
% distributed under the License is distributed on an "AS IS" BASIS,
% WITHOUT WARRANTIES OR CONDITIONS OF ANY KIND, either express or implied.
% See the License for the specific language governing permissions and
% limitations under the License.

%\documentclass[12pt]{minimal}
\documentclass[12pt]{scrartcl}
\usepackage[margin=30mm,nofoot,pdftex,a4paper,landscape]{geometry}
\usepackage[pdftex,inactive,tightpage]{preview}
\usepackage[table,dvipsnames,svgnames]{xcolor} % loads also »colortbl«
\usepackage[T1]{fontenc} % needed for fonts
\usepackage{attachfile}
\usepackage{array}
\usepackage{multicol}

% The PEM files attached herein can be created by simply copying the
% smime_keys-produced certificate and renaming, or exporting a pkcs12
% (.p12) file from the browser and extracting the certificate:
% 
%   $ openssl pkcs12 -in key_pair.p12 -clcerts -nokeys -out pubkey.pem

\newcommand{\dn}{@gombos.info}
\newcommand{\pdn}{@protonmail.ch}

\input{work/pubkeys_files.tex} % <= read auto-generated file to define commands for \pgplegal, \pgpwork, etc.

\newcommand{\friendetal}{friend\textsuperscript{(or f.o.f)}/ comrade/ colleague/ classmate/ VIP/ other}

\newcommand{\protonfootnote}{%
  Importing the protonmail key makes it possible to direct mail to
  Justin's protonmail address from outside protonmail.  This is rarely
  sensible though because the \dn\ addresses have a faster response
  time and better filtering.}

\begin{document}
\begin{preview}
  \section{Justin's public keys}
  \begin{multicols}{2}
    The right two columns of the following table have attachments to
    Justin's public keys.  You need to know whether your mail client
    uses S/MIME or PGP.  If you forgot, look back at the
    \href{https://alchemistanaut.github.io/}{diagram}.  Clicking on an
    attachment may automatically import it, depending on your the MIME
    application associations in your environment.  If not, you can
    right-click, save to a file, then import the key manually.
  \end{multicols}

  \rowcolors{2}{Khaki}{LightGoldenrodYellow}
  \begin{minipage}{6cm}
  \setlength{\tabcolsep}{6pt}%this should be the default - not sure why it must be explicit
%  \begin{tabular}{l>{\ttfamily}r>{\fontsize{1mm}{2mm}\selectfont}rr}
  \begin{tabular}{>{\raggedright}p{7cm}>{\ttfamily}r>{\tiny}rc}
    \textsl{\textbf{if e-mailing as a(n)..}}%
    & \multicolumn{1}{r}{\textsl{\textbf{then address to}}}%
    & \multicolumn{1}{r}{\textsl{\textbf{using \underline{S/MIME certificate}}}}%
    & \textsl{\textbf{or \underline{PGP key} (binary)}}\\
    accountant/lawyer           & \href{mailto:legal\dn}{legal\dn}   & e-mail support\dn\ for a key & \pgplegal\\
    client/professor/recruiter  & \href{mailto:work\dn}{work\dn}     & e-mail support\dn\ for a key & \pgpwork\\
    \friendetal                 & \href{mailto:crypto\dn}{crypto\dn} & \certcrypto                  & \pgpcrypto\\
    protonmail sender\footnote{\protonfootnote} & \href{mailto:bletchleylabs\pdn}{bletchleylabs\pdn} & n/a & \pgpproton\\
    stranger                    & \href{mailto:public\dn}{public\dn} & e-mail support\dn\ for a key & \pgppublic\\
  \end{tabular}
\end{minipage}

\section{Trivia: public PDF password rationale}
\begin{multicols}{2}
  Some will wonder why this PDF is encrypted while the password is
  publicly shared right next to the file.  The public keys contained
  in this PDF file all have metadata that contains e-mail addresses
  which could be easily bot-harvested if exposed, and consequently
  spammed.  So the idea is that it would be unlikely for a bot to be
  clever enough to capture passwords and then associate them to PDF
  documents.  Notice also that the e-mail addresses in the table are
  active links written in a proper font.
\end{multicols}

\end{preview}
\end{document}
